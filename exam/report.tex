\documentclass[10pt]{article}
\usepackage{graphicx}
\usepackage{mathtools}
\usepackage{listings}

\begin{document}
	\section*{Questions 7}
	\begin{enumerate}
		\item How do you allocate GSL vectors and matrices?
		
		To allocate a GSL vectors, you type:

		\texttt{gsl\_vector *v1 = gsl\_vector\_alloc (size\_t n);}

		\texttt{gsl\_vector *v2 = gsl\_vector\_calloc (size\_t n);}

		To allocate a GSL matrix, you type:

		\texttt{gsl\_matrix *m1 = gsl\_vector\_alloc (size\_t n1, size\_t n2);}

		\texttt{gsl\_matrix *m2 = gsl\_vector\_calloc (size\_t n1, size\_t n2);}

		the difference between \texttt{alloc} and \texttt{calloc}, is that \texttt{calloc} initializes all the elements as zeros, where \texttt{alloc} doesn't.
		In the end, you will need to free them, with

		\texttt{gsl\_vector\_free (gsl\_vector *v);}

		\texttt{gsl\_matrix\_free (gsl\_vector *m);}
		
	\end{enumerate}
	\section*{Problem 20: Implement the Arctangent function using integral representation.}
	\begin{align}
		\mathrm{arctan}(x) 
		= \int_{0}^{x} \frac{1}{z^2+1} \, dz. \label{eq:integration_func}
	\end{align}
	To facilitate numerical integration reduce the argument to a reasonable interval (e.g. $[0,1]$) using the formulae (check them),
	\begin{align}
		\mathrm{arctan}(-x) = -\mathrm{arctan}(x) \\
		\mathrm{arctan}\!\left(\frac{1}{x}\right) =
		\frac{\pi}{2} - \mathrm{arctan}(x), \ \text{ if } x > 0.
	\end{align}
	prior to integration. Compare with the corresponding function from \texttt{<math.h>} or from \texttt{GSL}.
	
	The problem was solved by integrating with \texttt{gsl/gsl\_integratson.h}. First the integrand from eq.~\ref{eq:integration_func} was define:
	\lstinputlisting[language=C, firstline=6, lastline=8]{my_math.c}
	 and then the GSL integration rutine was build:
	\lstinputlisting[language=C, firstline=10]{my_math.c}
	The three if statements ensures that the function only integrates in the range of $[0,1]$, sence arctan converges to $\frac{\pi}{2}$ for $x \to \infty$ and $-\frac{\pi}{2}$ for $x \to -\infty$.
	
	The result of the integration was compared using \texttt{atan(x)} from \texttt{<math.h>} in the main function, and plottet with points, see fig.~\ref{fig:arctan_plot}.

	\begin{figure}[hb]
	\centering
	% GNUPLOT: LaTeX picture
\setlength{\unitlength}{0.240900pt}
\ifx\plotpoint\undefined\newsavebox{\plotpoint}\fi
\sbox{\plotpoint}{\rule[-0.200pt]{0.400pt}{0.400pt}}%
\begin{picture}(1500,900)(0,0)
\sbox{\plotpoint}{\rule[-0.200pt]{0.400pt}{0.400pt}}%
\put(130.0,82.0){\rule[-0.200pt]{4.818pt}{0.400pt}}
\put(110,82){\makebox(0,0)[r]{$-1$}}
\put(1419.0,82.0){\rule[-0.200pt]{4.818pt}{0.400pt}}
\put(130.0,160.0){\rule[-0.200pt]{4.818pt}{0.400pt}}
\put(110,160){\makebox(0,0)[r]{$-0.8$}}
\put(1419.0,160.0){\rule[-0.200pt]{4.818pt}{0.400pt}}
\put(130.0,237.0){\rule[-0.200pt]{4.818pt}{0.400pt}}
\put(110,237){\makebox(0,0)[r]{$-0.6$}}
\put(1419.0,237.0){\rule[-0.200pt]{4.818pt}{0.400pt}}
\put(130.0,315.0){\rule[-0.200pt]{4.818pt}{0.400pt}}
\put(110,315){\makebox(0,0)[r]{$-0.4$}}
\put(1419.0,315.0){\rule[-0.200pt]{4.818pt}{0.400pt}}
\put(130.0,393.0){\rule[-0.200pt]{4.818pt}{0.400pt}}
\put(110,393){\makebox(0,0)[r]{$-0.2$}}
\put(1419.0,393.0){\rule[-0.200pt]{4.818pt}{0.400pt}}
\put(130.0,471.0){\rule[-0.200pt]{4.818pt}{0.400pt}}
\put(110,471){\makebox(0,0)[r]{$0$}}
\put(1419.0,471.0){\rule[-0.200pt]{4.818pt}{0.400pt}}
\put(130.0,548.0){\rule[-0.200pt]{4.818pt}{0.400pt}}
\put(110,548){\makebox(0,0)[r]{$0.2$}}
\put(1419.0,548.0){\rule[-0.200pt]{4.818pt}{0.400pt}}
\put(130.0,626.0){\rule[-0.200pt]{4.818pt}{0.400pt}}
\put(110,626){\makebox(0,0)[r]{$0.4$}}
\put(1419.0,626.0){\rule[-0.200pt]{4.818pt}{0.400pt}}
\put(130.0,704.0){\rule[-0.200pt]{4.818pt}{0.400pt}}
\put(110,704){\makebox(0,0)[r]{$0.6$}}
\put(1419.0,704.0){\rule[-0.200pt]{4.818pt}{0.400pt}}
\put(130.0,781.0){\rule[-0.200pt]{4.818pt}{0.400pt}}
\put(110,781){\makebox(0,0)[r]{$0.8$}}
\put(1419.0,781.0){\rule[-0.200pt]{4.818pt}{0.400pt}}
\put(130.0,859.0){\rule[-0.200pt]{4.818pt}{0.400pt}}
\put(110,859){\makebox(0,0)[r]{$1$}}
\put(1419.0,859.0){\rule[-0.200pt]{4.818pt}{0.400pt}}
\put(130.0,82.0){\rule[-0.200pt]{0.400pt}{4.818pt}}
\put(130,41){\makebox(0,0){$0$}}
\put(130.0,839.0){\rule[-0.200pt]{0.400pt}{4.818pt}}
\put(261.0,82.0){\rule[-0.200pt]{0.400pt}{4.818pt}}
\put(261,41){\makebox(0,0){$2$}}
\put(261.0,839.0){\rule[-0.200pt]{0.400pt}{4.818pt}}
\put(392.0,82.0){\rule[-0.200pt]{0.400pt}{4.818pt}}
\put(392,41){\makebox(0,0){$4$}}
\put(392.0,839.0){\rule[-0.200pt]{0.400pt}{4.818pt}}
\put(523.0,82.0){\rule[-0.200pt]{0.400pt}{4.818pt}}
\put(523,41){\makebox(0,0){$6$}}
\put(523.0,839.0){\rule[-0.200pt]{0.400pt}{4.818pt}}
\put(654.0,82.0){\rule[-0.200pt]{0.400pt}{4.818pt}}
\put(654,41){\makebox(0,0){$8$}}
\put(654.0,839.0){\rule[-0.200pt]{0.400pt}{4.818pt}}
\put(785.0,82.0){\rule[-0.200pt]{0.400pt}{4.818pt}}
\put(785,41){\makebox(0,0){$10$}}
\put(785.0,839.0){\rule[-0.200pt]{0.400pt}{4.818pt}}
\put(915.0,82.0){\rule[-0.200pt]{0.400pt}{4.818pt}}
\put(915,41){\makebox(0,0){$12$}}
\put(915.0,839.0){\rule[-0.200pt]{0.400pt}{4.818pt}}
\put(1046.0,82.0){\rule[-0.200pt]{0.400pt}{4.818pt}}
\put(1046,41){\makebox(0,0){$14$}}
\put(1046.0,839.0){\rule[-0.200pt]{0.400pt}{4.818pt}}
\put(1177.0,82.0){\rule[-0.200pt]{0.400pt}{4.818pt}}
\put(1177,41){\makebox(0,0){$16$}}
\put(1177.0,839.0){\rule[-0.200pt]{0.400pt}{4.818pt}}
\put(1308.0,82.0){\rule[-0.200pt]{0.400pt}{4.818pt}}
\put(1308,41){\makebox(0,0){$18$}}
\put(1308.0,839.0){\rule[-0.200pt]{0.400pt}{4.818pt}}
\put(1439.0,82.0){\rule[-0.200pt]{0.400pt}{4.818pt}}
\put(1439,41){\makebox(0,0){$20$}}
\put(1439.0,839.0){\rule[-0.200pt]{0.400pt}{4.818pt}}
\put(130.0,82.0){\rule[-0.200pt]{0.400pt}{187.179pt}}
\put(130.0,82.0){\rule[-0.200pt]{315.338pt}{0.400pt}}
\put(1439.0,82.0){\rule[-0.200pt]{0.400pt}{187.179pt}}
\put(130.0,859.0){\rule[-0.200pt]{315.338pt}{0.400pt}}
\put(1279,818){\makebox(0,0)[r]{data}}
\put(130,471){\makebox(0,0){$+$}}
\put(137,509){\makebox(0,0){$+$}}
\put(143,548){\makebox(0,0){$+$}}
\put(150,585){\makebox(0,0){$+$}}
\put(156,622){\makebox(0,0){$+$}}
\put(163,657){\makebox(0,0){$+$}}
\put(169,690){\makebox(0,0){$+$}}
\put(176,721){\makebox(0,0){$+$}}
\put(182,749){\makebox(0,0){$+$}}
\put(189,775){\makebox(0,0){$+$}}
\put(195,797){\makebox(0,0){$+$}}
\put(202,817){\makebox(0,0){$+$}}
\put(209,833){\makebox(0,0){$+$}}
\put(215,845){\makebox(0,0){$+$}}
\put(222,853){\makebox(0,0){$+$}}
\put(228,858){\makebox(0,0){$+$}}
\put(235,859){\makebox(0,0){$+$}}
\put(241,856){\makebox(0,0){$+$}}
\put(248,849){\makebox(0,0){$+$}}
\put(254,838){\makebox(0,0){$+$}}
\put(261,824){\makebox(0,0){$+$}}
\put(267,806){\makebox(0,0){$+$}}
\put(274,785){\makebox(0,0){$+$}}
\put(281,760){\makebox(0,0){$+$}}
\put(287,733){\makebox(0,0){$+$}}
\put(294,703){\makebox(0,0){$+$}}
\put(300,671){\makebox(0,0){$+$}}
\put(307,637){\makebox(0,0){$+$}}
\put(313,601){\makebox(0,0){$+$}}
\put(320,563){\makebox(0,0){$+$}}
\put(326,525){\makebox(0,0){$+$}}
\put(333,487){\makebox(0,0){$+$}}
\put(339,448){\makebox(0,0){$+$}}
\put(346,409){\makebox(0,0){$+$}}
\put(353,371){\makebox(0,0){$+$}}
\put(359,334){\makebox(0,0){$+$}}
\put(366,299){\makebox(0,0){$+$}}
\put(372,265){\makebox(0,0){$+$}}
\put(379,233){\makebox(0,0){$+$}}
\put(385,203){\makebox(0,0){$+$}}
\put(392,176){\makebox(0,0){$+$}}
\put(398,153){\makebox(0,0){$+$}}
\put(405,132){\makebox(0,0){$+$}}
\put(411,115){\makebox(0,0){$+$}}
\put(418,101){\makebox(0,0){$+$}}
\put(425,91){\makebox(0,0){$+$}}
\put(431,84){\makebox(0,0){$+$}}
\put(438,82){\makebox(0,0){$+$}}
\put(444,83){\makebox(0,0){$+$}}
\put(451,89){\makebox(0,0){$+$}}
\put(457,98){\makebox(0,0){$+$}}
\put(464,111){\makebox(0,0){$+$}}
\put(470,127){\makebox(0,0){$+$}}
\put(477,147){\makebox(0,0){$+$}}
\put(483,170){\makebox(0,0){$+$}}
\put(490,196){\makebox(0,0){$+$}}
\put(497,225){\makebox(0,0){$+$}}
\put(503,257){\makebox(0,0){$+$}}
\put(510,290){\makebox(0,0){$+$}}
\put(516,325){\makebox(0,0){$+$}}
\put(523,362){\makebox(0,0){$+$}}
\put(529,400){\makebox(0,0){$+$}}
\put(536,438){\makebox(0,0){$+$}}
\put(542,477){\makebox(0,0){$+$}}
\put(549,516){\makebox(0,0){$+$}}
\put(555,554){\makebox(0,0){$+$}}
\put(562,592){\makebox(0,0){$+$}}
\put(569,628){\makebox(0,0){$+$}}
\put(575,662){\makebox(0,0){$+$}}
\put(582,695){\makebox(0,0){$+$}}
\put(588,726){\makebox(0,0){$+$}}
\put(595,754){\makebox(0,0){$+$}}
\put(601,779){\makebox(0,0){$+$}}
\put(608,801){\makebox(0,0){$+$}}
\put(614,820){\makebox(0,0){$+$}}
\put(621,835){\makebox(0,0){$+$}}
\put(627,847){\makebox(0,0){$+$}}
\put(634,854){\makebox(0,0){$+$}}
\put(641,858){\makebox(0,0){$+$}}
\put(647,859){\makebox(0,0){$+$}}
\put(654,855){\makebox(0,0){$+$}}
\put(660,847){\makebox(0,0){$+$}}
\put(667,836){\makebox(0,0){$+$}}
\put(673,821){\makebox(0,0){$+$}}
\put(680,803){\makebox(0,0){$+$}}
\put(686,781){\makebox(0,0){$+$}}
\put(693,756){\makebox(0,0){$+$}}
\put(699,728){\makebox(0,0){$+$}}
\put(706,698){\makebox(0,0){$+$}}
\put(713,665){\makebox(0,0){$+$}}
\put(719,631){\makebox(0,0){$+$}}
\put(726,594){\makebox(0,0){$+$}}
\put(732,557){\makebox(0,0){$+$}}
\put(739,519){\makebox(0,0){$+$}}
\put(745,480){\makebox(0,0){$+$}}
\put(752,441){\makebox(0,0){$+$}}
\put(758,403){\makebox(0,0){$+$}}
\put(765,365){\makebox(0,0){$+$}}
\put(771,328){\makebox(0,0){$+$}}
\put(778,293){\makebox(0,0){$+$}}
\put(785,259){\makebox(0,0){$+$}}
\put(791,228){\makebox(0,0){$+$}}
\put(798,199){\makebox(0,0){$+$}}
\put(804,172){\makebox(0,0){$+$}}
\put(811,149){\makebox(0,0){$+$}}
\put(817,129){\makebox(0,0){$+$}}
\put(824,112){\makebox(0,0){$+$}}
\put(830,99){\makebox(0,0){$+$}}
\put(837,89){\makebox(0,0){$+$}}
\put(843,84){\makebox(0,0){$+$}}
\put(850,82){\makebox(0,0){$+$}}
\put(856,84){\makebox(0,0){$+$}}
\put(863,90){\makebox(0,0){$+$}}
\put(870,100){\makebox(0,0){$+$}}
\put(876,113){\makebox(0,0){$+$}}
\put(883,130){\makebox(0,0){$+$}}
\put(889,151){\makebox(0,0){$+$}}
\put(896,174){\makebox(0,0){$+$}}
\put(902,201){\makebox(0,0){$+$}}
\put(909,230){\makebox(0,0){$+$}}
\put(915,262){\makebox(0,0){$+$}}
\put(922,296){\makebox(0,0){$+$}}
\put(928,331){\makebox(0,0){$+$}}
\put(935,368){\makebox(0,0){$+$}}
\put(942,406){\makebox(0,0){$+$}}
\put(948,445){\makebox(0,0){$+$}}
\put(955,484){\makebox(0,0){$+$}}
\put(961,522){\makebox(0,0){$+$}}
\put(968,560){\makebox(0,0){$+$}}
\put(974,598){\makebox(0,0){$+$}}
\put(981,634){\makebox(0,0){$+$}}
\put(987,668){\makebox(0,0){$+$}}
\put(994,701){\makebox(0,0){$+$}}
\put(1000,731){\makebox(0,0){$+$}}
\put(1007,758){\makebox(0,0){$+$}}
\put(1014,783){\makebox(0,0){$+$}}
\put(1020,804){\makebox(0,0){$+$}}
\put(1027,822){\makebox(0,0){$+$}}
\put(1033,837){\makebox(0,0){$+$}}
\put(1040,848){\makebox(0,0){$+$}}
\put(1046,855){\makebox(0,0){$+$}}
\put(1053,859){\makebox(0,0){$+$}}
\put(1059,858){\makebox(0,0){$+$}}
\put(1066,854){\makebox(0,0){$+$}}
\put(1072,846){\makebox(0,0){$+$}}
\put(1079,834){\makebox(0,0){$+$}}
\put(1086,818){\makebox(0,0){$+$}}
\put(1092,799){\makebox(0,0){$+$}}
\put(1099,777){\makebox(0,0){$+$}}
\put(1105,751){\makebox(0,0){$+$}}
\put(1112,723){\makebox(0,0){$+$}}
\put(1118,692){\makebox(0,0){$+$}}
\put(1125,659){\makebox(0,0){$+$}}
\put(1131,625){\makebox(0,0){$+$}}
\put(1138,588){\makebox(0,0){$+$}}
\put(1144,551){\makebox(0,0){$+$}}
\put(1151,512){\makebox(0,0){$+$}}
\put(1158,474){\makebox(0,0){$+$}}
\put(1164,435){\makebox(0,0){$+$}}
\put(1171,396){\makebox(0,0){$+$}}
\put(1177,359){\makebox(0,0){$+$}}
\put(1184,322){\makebox(0,0){$+$}}
\put(1190,287){\makebox(0,0){$+$}}
\put(1197,254){\makebox(0,0){$+$}}
\put(1203,223){\makebox(0,0){$+$}}
\put(1210,194){\makebox(0,0){$+$}}
\put(1216,168){\makebox(0,0){$+$}}
\put(1223,145){\makebox(0,0){$+$}}
\put(1230,126){\makebox(0,0){$+$}}
\put(1236,110){\makebox(0,0){$+$}}
\put(1243,97){\makebox(0,0){$+$}}
\put(1249,88){\makebox(0,0){$+$}}
\put(1256,83){\makebox(0,0){$+$}}
\put(1262,82){\makebox(0,0){$+$}}
\put(1269,85){\makebox(0,0){$+$}}
\put(1275,91){\makebox(0,0){$+$}}
\put(1282,102){\makebox(0,0){$+$}}
\put(1288,116){\makebox(0,0){$+$}}
\put(1295,134){\makebox(0,0){$+$}}
\put(1302,155){\makebox(0,0){$+$}}
\put(1308,179){\makebox(0,0){$+$}}
\put(1315,206){\makebox(0,0){$+$}}
\put(1321,236){\makebox(0,0){$+$}}
\put(1328,268){\makebox(0,0){$+$}}
\put(1334,302){\makebox(0,0){$+$}}
\put(1341,337){\makebox(0,0){$+$}}
\put(1347,375){\makebox(0,0){$+$}}
\put(1354,413){\makebox(0,0){$+$}}
\put(1360,451){\makebox(0,0){$+$}}
\put(1367,490){\makebox(0,0){$+$}}
\put(1374,529){\makebox(0,0){$+$}}
\put(1380,567){\makebox(0,0){$+$}}
\put(1387,604){\makebox(0,0){$+$}}
\put(1393,640){\makebox(0,0){$+$}}
\put(1400,674){\makebox(0,0){$+$}}
\put(1406,706){\makebox(0,0){$+$}}
\put(1413,735){\makebox(0,0){$+$}}
\put(1419,762){\makebox(0,0){$+$}}
\put(1426,787){\makebox(0,0){$+$}}
\put(1432,808){\makebox(0,0){$+$}}
\put(1349,818){\makebox(0,0){$+$}}
\put(130.0,82.0){\rule[-0.200pt]{0.400pt}{187.179pt}}
\put(130.0,82.0){\rule[-0.200pt]{315.338pt}{0.400pt}}
\put(1439.0,82.0){\rule[-0.200pt]{0.400pt}{187.179pt}}
\put(130.0,859.0){\rule[-0.200pt]{315.338pt}{0.400pt}}
\end{picture}

	\caption{Plot of calculated and exact arctan(x), every 10th result of the arctan from math.h was plottet as points for clarity.}
	\label{fig:arctan_plot}
	\end{figure}
\end{document}
